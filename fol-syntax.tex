\section{Sintaxe}

    A sintaxe de um sistema lógico aborda basicamente os símbolos que são utilizados para representá-lo. Portanto, nesta seção, serão abordados funções, predicados, relações e quantificadores, dentre eles $\forall$ e $\exists$.

    \subsection{Funções, Predicados e Relações}

        Dentro de Lógica de Primeira Ordem, funções, predicados e relações são mapeamentos que, dado algum elemento do domínio, retornam uma proposição ou outro elemento do domínio. Parece confuso? Vamos olhar um exemplo.

        \begin{center}
            $x$ natural é par ou ímpar.            
        \end{center}

        Nosso domínio, neste caso, são os números naturais ($\mathbb{N}$). Dizemos que \textit{par} é "algo" que recebe um número e retorna $V$ ou $F$. Chamamos isso de \textbf{predicado}. Sendo assim, podemos escrever:

        \begin{center}
            $par(x) \lor impar(x)$.
        \end{center}

        A partir deste exemplo, podemos extrair alguns símbolos para exemplificar a construção de um sistema lógico de primeira ordem.

        \begin{itemize}
            \item O domínio são os naturais;
            \item Os objetos são os números $0$, $1$, $2$ etc.;
            \item Existem \textbf{funções}, como \textit{adição} e \textit{subtração}, que recebem (zero ou mais) números e retornam outros números;
            \item Existem \textbf{predicados}, como \textit{par} e \textit{ímpar}, que recebem um número e retornam $V$ ou $F$;
            \item Existem \textbf{relações}, como \textit{igual} e \textit{menor}, que recebem dois números e retornam $V$ ou $F$.
        \end{itemize}

        Os objetos pertencentes ao domínio, chamados constantes, como o $1$ e o $4$, no exemplo, podem ser considerados funções que tomam zero elementos. Além disso, podemos considerar predicados que tomam zero elementos como os valores lógicos $\top$ e $\bot$.

        Expressões que representam elementos do domínio(incluindo funções de elementos) são chamados de \textbf{termos}. Alguns exemplos:

        \begin{itemize}
            \item $5$
            \item $sucessor(10)$ (função sucessora, retorna o elemento acrescido de $1$)
            \item $33+44$
        \end{itemize}

        Observe que o símbolo para a função de adição está "infixo"; poderíamos ter representado como $+(33, 44)$ ou $adicao(33, 44)$.
        
        Expressões que retornam $V$ ou $F$ são chamadas \textbf{fórmulas}:

        \begin{itemize}
            \item $maiorDoQue(1,2)$ ($1 > 2$)
            \item $par(10) \lor impar(5)$
            \item $2=2$
        \end{itemize}

        O Lean é muito eficiente em expressar Lógica de Primeira Ordem. Vejamos nosso exemplo:

        \begin{lstlisting}
constant U : Type
constant zero : U
constant par : U → Prop
constant primo : U → Prop
constant igual : U → U → Prop
constant adicao : U → U → U
\end{lstlisting}

        Pelo fato de que o Lean é baseado em \textit{Teoria dos Tipos}, declaramos um novo tipo \lstinline{U}. Podemos intuitivamente considerá-lo como "universo" ou "domínio". Por exemplo, o conjunto dos naturais.

        Foi declarado um objeto chamado \lstinline{zero} do tipo \lstinline{U} (nossa analogia com o zero natural).
        \lstinline{par} é um predicado, pois toma um elemento do tipo \lstinline{U} e retona um elemento do tipo proposição (\lstinline{Prop}).

        \lstinline{adicao} é uma função que toma dois elementos do tipo \lstinline{U} e retorna outro do mesmo tipo. Ora, podemos constatar:

        \begin{lstlisting}
#check par zero
#check adicao zero zero
#check par (adicao zero zero)
\end{lstlisting}

        O \lstinline{#check} da linha $1$ informa que a expressão tem tipo \lstinline{Prop}; na linha $2$, tipo \lstinline{U}; e, na linha $3$, tipo \lstinline{Prop}. Importante observar o papel dos parênteses acima, para que \lstinline{par} receba apenas um elemento.

        Uma função (ou relação) que recebe mais de um elemento tem notação \lstinline{U → U → U}. A notação para predicados (\lstinline{U → Prop}) e relações (\lstinline{U → U → Prop}) funciona como se ambas fossem funções, porém retornassem \lstinline{Prop}. 

        Vários conjuntos estão nas bibliotecas padrão do Lean, como os naturais, utilizados nos exemplos anteriores. O comando para o símbolo \lstinline{ℕ} é \lstinline{\nat} ou \lstinline{\N}.

        \begin{lstlisting}
constant zero : ℕ
#check zero + zero
\end{lstlisting}

        O \lstinline{#check} da linha $2$ retorna algo do tipo \lstinline{ℕ}.

        Podemos misturar Lógica Proposicional com Lógica de Primeira Ordem:

        \begin{lstlisting}
constant U : Type
constant zero : U
constant par : U → Prop
constant primo : U → Prop
constant igual : U → U → Prop
constant adicao : U → U → U

#check ¬ (par zero ∨ par (adicao zero zero)) ∧ primo zero
\end{lstlisting}

        E o \lstinline{#check} nos retorna algo do tipo \lstinline{Prop}.

    \subsection{Quantificador Universal}

        Grande parte do poder da Lógica Proposicional se deve aos quantificadores.
        O símbolo $\forall$ é o quantificador universal, que representa "para todo".
        Quando ele é seguido de uma variável e de uma expressão, ele indica que aquela expressão é verdadeira para toda variável do domínio.
        Por exemplo:

        \begin{itemize}
            \item $\forall x, (par(x) \lor impar(x))$
            \item $\forall y, (par(y) \rightarrow impar(y + 1))$
        \end{itemize}

        A primeira expressão nos diz que todo número é par ou ímpar (no caso dos naturais).
        A segunda diz que, para todo número, o fato dele ser par implica que seu sucessor é ímpar.
        Vejamos essas expressões no Lean:

        \begin{lstlisting}
constant U : Type
constant par : U → Prop
constant impar : U → Prop
constant sucessor : U → U

#check ∀ x : U, par x ∨ impar x
#check ∀ y : U, par y → impar (sucessor y)
\end{lstlisting}

        Os dois \lstinline{#check}'s nos retornam \lstinline{Prop}.

        Observe as três sentenças:

        \begin{itemize}
            \item $\forall x, (par(x) \lor impar(x))$
            \item $\forall x, par(x) \lor impar(x)$
            \item $\forall x, (par(x)) \lor impar(x)$
        \end{itemize}

        Por uma questão de convenção, as duas últimas sentenças são equivalentes, enquanto a primeira é diferente.
        Neste contexto, estamos lidando com o \textbf{escopo} da variável $x$. A convenção diz, portanto, que o escopo da variável é o menor possível.
        
        Curiosamente, o modo como o Lean lida com escopo é diferente: \lstinline{∀ x : U, par x ∨ impar x} equivale a \lstinline{∀ x : U, (par x ∨ impar x)}.
        Ou seja, o Lean busca o maior escopo possível.

        Quando estamos lidando com quantificadores, a variável que o acompanha é dita \textbf{limitada} (\textit{bound}, em inglês).
        na expressão $\forall x, A(x)$, a variável $x$ é limitada.
        Isso significa que o $x$ não representa um valor em si, mas apenas um ``espaço reservado"\space para qualquer outra variável.
        Observe que a expressão $\forall y, A(y)$ representa exatamente a mesma coisa.

    \subsection{Quantificador Existencial}