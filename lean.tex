\chapter{Introdução ao Lean }
% https://leanprover.github.io/theorem_proving_in_lean/introduction.html

Ao final do último capítulo, definimos duas classes fundamentais de ferramentas de auxílio a prova de teoremas, os \textit{ATPs} e \textit{ITPs}.
Esses grupos de ferramentas em geral são muito bem definidos, e distanciados, mantendo um espaço vazio entre si, o que pode parecer desvantajoso:o  matemático pode desejar algo entre a prova automática e interativa, ou o acesso a ambos os recursos numa mesma ferramenta.

O provador de teoremas Lean\footnote{documentação, ambiente e tutoriais estão disponiveis em: leanprover.github.io} busca preencher essa lacuna: uma única ferramenta contendo o melhor de ambos os mundos acessíveis.
Nesse sentido, age como uma \textit{ponte entre os dois mundos}.

O ambiente do Lean suporta interação, construção de termos, e verificação passo-a-passo das expressões desenvolvidas.
Lean implementa o chamado \textit{calculus of constructions}, ou cálculo de contruções, um sistema dedutivo bastante poderoso, que contém os demais sistemas a serem discutidos ao longo do livro, tais como Lógica de Proposições, ou Lógica de Primeira Ordem.

Um usuário inicial pode se deparar com algumas dúvidas pertinentes: \textit{porque contruir um termo é uma prova?}, \textit{consigo desenvolver provas top-down e botton-up?}, ou \textit{como conhecer as bibliotecas que o Lean já tem implementadas?}Algumas dessas dúvidas são esclarecidas logo mais, enquanto outras serão deixadas para pŕatica ao longo dos proximos capítulos.

\section{Teoria dos tipos}
% https://leanprover.github.io/theorem_proving_in_lean/dependent_type_theory.html
% https://en.wikipedia.org/wiki/Type_theory
\textit{Teoria dos Tipos} é uma classe de linguagens poderosa, que permite desenvolver assertivas bem definidas sobre definições matemáticas, ou exprimir conceitos desde os abstratos aos mais paupáveis.
De fato, essa é uma linguagem bastante expressiva, e útil para definição de elementos formais; até certo ponto, é semelhante a noção de teoria dos conjuntos.

No entanto, em muitos contextos, é razoável utilizar uma linguagem com ferramental teórico capaz de lidar com uma gama de elementos distitos, pertencentes a diversas classes, \textit{tipos distintos}.
Isso ocorre frequentemente quando tratamos de objetos matemáticos, e formalismos nesse sentido: o elemento $1$ pode ter o tipo natural ou real, assim como $[1,2,3, ...]$ pode expressar o tipo de uma lista de inteiros, dependendo da interpretação e contexto dados.
Claramente lidamos constantemente com elementos de tipos distintos, objetos que se relacionam, e tipos que se relacionam.
Pra isso, \textit{teoria dos tipos} se mostra extremamante útil.

Lean, em especial, implementa uma versão da linguagem chamada \textit{Calculo das Construções}, ou \textit{Calculus of Constructions}. A partir dos próximos exemplos, o leitor é convidado a acompanhar fazendo a intelação da ferramenta, ou através do \textit{Lean Web Editor\footnote{leanprover.github.io/live/latest/}}.

\subsection{Noções Fundamentais}
Em teoria dos tipos, tudo tem um tipo definido. Essa é uma exigência razoável; de fato, somos capazes de \textit{tipar} qualquer elemento dentro de um contexto definido, em uma linguagem adequada. Vejamos alguns exemplos em Lean:

\vspace{5mm}
\begin{lstlisting}
    -- abaixo declaramos algumas variaveis e usamos o
    -- comando #check para checar o tipo do elemento
    variable b : bool
    variable n : ℕ
    variable f : ℕ → ℕ
    variable g : ℕ × ℕ → ℕ

    #check b            -- b : bool
    #check 1            -- 1 : ℕ
    #check n + 1        -- n + 1 : ℕ

    #check f n          -- f n : ℕ
    #check g (1,n)      -- g (1, n) : ℕ
\end{lstlisting}
\vspace{5mm}

\noindent O comando \textit{variable} declara uma nova variável no contexto.
A partir disso, somos capazes de checar os tipo.

Note nas linhas $5$ e $6$ que $f$ e $g$ podem ser interpretadas como funções de naturais em naturais, então pudemos fazer a aplicação em $12$ e $13$, recebendo $f n$ e $g (1,n)$ como naturais.
Essa saída faz sentido em termos de tipo: sabemos que a aplicação tem tipo natural, embora não saibamos \textit{qual} natural.

Um leitor curioso pode questionar \textit{se tudo tem um tipo, qual tipo dos tipos?} ou ainda ir além, e \textit{qual o tipo do tipo dos tipos?} E essas perguntas podem ser respondidas novamente através do Lean:

\vspace{5mm}
\begin{lstlisting}
    #check ℕ            -- ℕ : Type
    #check bool         -- bool : Type

    -- considere Type = Type 0
    #check Type 0       -- Type : Type 1
    #check Type 1       -- Type : Type 2
    #check Type 2       -- Type : Type 3
\end{lstlisting}
\vspace{5mm}

\noindent Essa pode ser uma resposta \textit{frustrante} ou \textit{brilhante}.
Os tipos básicos em Lean tem o tipo \textit{Type}, ou \textit{Type 0}, enquanto o proprio tipo \textit{Type 0} tem tipo \textit{Type 1} quem tem tipo \textit{Type 2} e assim por diante.

Essa noção de hierarquia de tipos é uma solução proposta inicialmente por Russel, e posteriormente desenvolvida por Chwistek e Ramsey, em sua \textit{teoria simples dos tipos}, para evitar o conflito entre a teoria dos conjuntos de Frege, e o paradoxo de Russel.
De forma simples, a hierarquia de tipos evita o autorreferenciamento, em que um elemento tem o próprio tipo.

O que o Lean se propõe a fazer, portanto, é oferencer um ambiente para tratar da álgebra de tipos, implementando conceitos como \textit{abstração} e \textit{aplicação}, ou \textit{tipos dependentes}.

\subsection{Proposições como Tipos}
% leanprover.github.io/theorem_proving_in_lean/propositions_and_proofs.html
Entramos em uma das discussões mais fundamentais sobre o \textit{como Lean provas um teorema}, e para isso, introduzimos a noção de \textit{Proposições como Tipos}, através do tipo \textit{Prop}.

Considere um tipo \textit{Prop} cujos habitantes (elementos daquele tipo) representam uma \textit{prova daquela proposição}.
Em outras palavras, dada uma proposição que se deseje provar, basta construirmos um termo \textit{habitante daquela proposição} através de aplicações, abstrações ou outras ténicas formais para lidar com os tipos.
Vejamoas um exemplo:

\vspace{5mm}
\begin{lstlisting}
    variables A B : Prop    -- A e B sao proposicoes

    variable h1 : A         -- h1 uma prova para A
    variable h2 : A → B     -- h2 uma prova para A → B

    example : B := h2 h1    -- h2 h1 prova para B
    #check h2 h1            -- h2 h1 : B
\end{lstlisting}
\vspace{5mm}

Acima, introduzimos sem maiores explicações algumas noções profundas sobre Lean, e lógica proposicional, que serão discutidas nos próximos capítulos.
No entanto, note que no contexto dado, fomos capazes de construir um termo \textit{h2 h1} que tem tipo \textit{B}, então costruimos uma prova para \textit{B}, na linha $6$; ainda podemos checar se esse termo tem o tipo desejado, em $7$.

Observe ainda, na linha $4$, em que criamos um elemento do tipo $A → B$, que anteriormente citamos como a noção de uma função que leva tipo $A$ em um tipo $B$.
Aqui, essa noção será extendida para a de \textit{implicação lógica}, tambem discutida no capítulo sobre lógica proposicional.

\section{Provas usando LEAN}
\begin{itemize}
    \item Exemplos (avancados) de provas utilizado LEAN.
    \item Podemos falar sobre o que é o tal "example" do item anterior.
    \item Talvez tambem falar sobre "theorem", e "lemma".
    \item Acho que "sections e namespaçes", não.
\end{itemize}

\subsection{Modo Termo}
Apresentação e exemplos

\subsection{Modo Tática}
Apresentação e exemplos

\subsection{Modo Cálculo}
Apresentação e exemplos

% será que faz sentido falarmos em aspectos praticos sobreo Lean, como onde ele está disponivel online, ou como conhecer os teormas basicos dispoiveis...
