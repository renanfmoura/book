\chapter{Conjuntos}

\section{Introdução}

    Para tratar de conjuntos iremos tentar uma abordagem que que não deixe de lado o rigor matemático necessário, mas que ao mesmo tempo permita utilizar o Lean e contextualizar exemplos do dia a dia. 
    
    Caro leitor, como você definiria conjunto? Vamos, pense um pouco. No século XIX, o matemático Georg Cantor, no jornal acadêmico Mathematische Annalen, descreveu um conjunto (ou utilizando sua terminologia, \textit{Menge}) como, em tradução livre: “Por conjunto nós entendemos qualquer coleção $M$ de objetos deteminados e distintos (chamados de elementos de $M$) da nossa intuição ou do nosso pensamento em um todo."
    
    Ou seja, mesmo um conjunto podendo ser algo tão abstrato quanto os números naturais ($\mathbb{N}$), os números reais ($\mathbb{R}$) e o Conjuntor de Cantor, podemos ter coisas menos abstratas como o conjunto das palavras desse texto, o dos planetas do Sistema Solar ou até dos alunos de sua turma. A questão é, todos esses conjuntos podem ser mais intuitivos ou abstratos (além de como são definidos) dependendo da pessoa que irá interpretá-los, por exemplo o conjunto dos naturais pode ser definido como $\mathbb{N} = \{1,2,3,...\}$ ou $\mathbb{N}=\{0,1,2,3,...\}$ (para não desagradar ninguém), o planetas do Sistema Solar $ P = \{ Mercurio, Venus, Terra, \\ Marte, Jupiter, Saturno, Urano, Netuno \}$ (lembramos que se este livro fosse escrito a uns 15 anos Plutão, na época, ainda seria considerado como um planeta, isto é, estaria no conjunto), por consequência, vemos que a interpretação do que determinado conjunto representa varia de pessoa para pessoa, mesmo que a ideia principal continue a mesma.
    
    Nosso intuito, durante essa aventura pelo mundo dos conjuntos será entender melhor certos conceitos e definições (como o fato do conjunto vazio estar contido em todos os conjuntos, ou se chegarmos até lá, porque o intervalo [0,1] nos reais não é enumerável), através de demosntrações, exemplos e exercícios, aumentando sua capacidade de abstração e a nossa também, já que para escrever esse capítulo nós teremos que ir além, pois não queremos apenas entender o que está aqui, mas que você entenda e aprenda também.

\section{Fundamentações}
    \subsection{Notações}
    Nesta seção iremos começar a introduzir notações matemáticas e algumas definições sobre conjuntos.
    
    \textbf{Pertence e Não Pertence:} Quando um determinado elemento $x$ faz parte de determinado conjunto $A$, nós dizemos que $x$ pertence a $A$ (denotamos $x \in A$). Caso $x$ não faça parte de $A$, diz-se que $x$ não pertence a $A$ (denota-se  $x \notin A$).
    
    \textbf{Contém, Contido e similares:} Já quando um conjunto $B$ possui todos os elementos que $A$ possui e, $B$ tem, pelo menos, um objeto que $A$ não possua, dizemos que $A$ está contido em $B$ ($\space A \subset B$) ou que $B$ contém $A$ ($B \supset A$)$^{\ref{fig:sets-02-00}}$. Se não sabemos se $B$ possui um objeto que $A$ não possua (e $B$ ainda possui todos os elementos que $A$), denotamos $A \subseteq B$, que quer dizer que $A \subset B$ ou $A=B$, de modo equivalente $B \supseteq A$, significa que $B \supset A$ ou $A=B$. Se $A$ possui pelo menos um elemento que $B$ não possua, dizemos que $A$ não está contido em $B$ ($A \not\subset B$) ou que $B$ não contém $B\not\supset A$, assim $A \neq B$. 
    
    \begin{figure}[hbt!]
        \centering      
        \includegraphics[width = 7 cm]{figures/sets/fig-sets-02-00.png}
        \caption{O conjunto $A$ está contido no conjunto $B$, equivalentemente, $B$ contém $A$.}
        \label{fig:sets-02-00}
    \end{figure}

    \textbf{Interseção:} Se estamos interessados em conjuntos/elementos que pertencem simultaneamente a dois conjuntos $A$ e $B$, dizemos que estamos interessados na interseção de $A$ e $B$ (denotada como $A \cap B$)$^{\ref{fig:sets-02-01}}$.
    
    \begin{figure}[hbt!]
        \centering      
        \includegraphics[width = 7 cm]{figures/sets/fig-sets-02-01.png}
        \caption{A região azul representa a visualização da interseção dos conjuntos $A$ e $B$.}
        \label{fig:sets-02-01}
    \end{figure}
    
    \textbf{União :}Já se estamos interessados nos conjuntos/elementos que fazem parte de $A$ ou de $B$ dizemos que, nosso objetivo é a união de $A$ e $B$ ($A \cup B$)$^{\ref{fig:sets-02-02}}$.
    
    \begin{figure}[hbt!]
        \centering      
        \includegraphics[width = 7 cm]{figures/sets/fig-sets-02-02.png}
        \caption{A região azul representa a visualização da união dos conjuntos $A$ e $B$.}
        \label{fig:sets-02-02}
    \end{figure}
    
    \textbf{Universo:} Quando estamos trabalhando com conjuntos é comum definirmos quem é nosso universo ($ \mathcal U $), isto é, o conjunto que conterá todos os conjuntos/elementos que estaremos trabalhando em um contexto$^{\ref{fig:sets-02-03}}$. Por exemplo, na reta real nosso universo é $\mathcal U = \mathbb{R}$.
    
    \begin{figure}[hbt!]
        \centering      
        \includegraphics[width = 7 cm]{figures/sets/fig-sets-02-03.png}
        \caption{A região azul representa a visualização do nosso Universo.}
        \label{fig:sets-02-03}
    \end{figure}
    
    \textbf{Conjunto Complementar:} Sendo $A$ um conjunto, dizemos que o conjunto $A$ complementar ou complemento de $A$ (denotado como $\overline A$ ou $A^C$)contém todos os conjuntos/elementos que não estão contidos/pertencem a $A$, mas fazem parte de nosso universo ($\mathcal U$)$^{\ref{fig:sets-02-04}}$.
    
    \begin{figure}[hbt!]
        \centering      
        \includegraphics[width = 7 cm]{figures/sets/fig-sets-02-04.png}
        \caption{A região azul representa a visualização do complemento de $A$.}
        \label{fig:sets-02-04}
    \end{figure}
    
    \textbf{Diferença de Conjuntos :} Quando temos dois conjuntos e nosso objetivo são os conjuntos/elementos que pertencem a um destes conjuntos, mas não do outro dizemos que estamos interessados na diferença destes conjuntos. No caso, se quero os conjuntos/elementos de $B$, mas não queremos pegar os que também pertencem a $A$, queremos os elementos/conjuntos que pertencem a diferença de $B$ com $A$ (denotamos como $B-A$ ou $B \backslash A$)$^{\ref{fig:sets-02-05}}$.
    
    \begin{figure}[hbt!]
        \centering      
        \includegraphics[width = 7 cm]{figures/sets/fig-sets-02-05.png}
        \caption{A região azul representa a visualização da diferença entre os conjuntos $B$ e $A$, isto é, a área onde estão os elementos que pertencem a $B$, mas não pertencem a $A$.}
        \label{fig:sets-02-05}
    \end{figure}
    
    Lembre-se: os diagramas apresentados servem como ferramenta auxiliar para ajudar a entender os conceitos, mas não devem ser vistos como a única ferramenta para compreender as definições e  a teoria exposta.
    
    \subsection{Axiomas}
    
    Agora iremos apresentar alguns axiomas que servirão como base para todo o desenvolvimento dos conteúdos aqui propostos. Quando utilizarmos a palavra elemento, estaremos utilizando-a com a ideia de que um conjunto que pertença a outro é um elemento do segundo, para evitar repetir o uso excessivo da palavra conjunto.
    
    \textbf{Axioma da Completude:} Dois conjuntos são iguais, se e somente se, todo conjunto que pertence ao primeiro conjunto pertence ao segundo e, todo conjunto que pertence ao segundo também pertence ao primeiro, ou seja:
    
    \[\forall a \hspace{1.5mm} \forall b \hspace{1.5mm} (a = b) \iff (\forall x \hspace{1.5mm} (x \in a \iff x \in b))\] 
    
    Através desse axioma fica mais claro de entender duas propriedades dos conjuntos. 
    
    Deste axioma, vem a explicação do motivo de que a ordem dos elementos de um conjunto não importa. Pois dado dois conjuntos com os mesmos elementos, mas em ordem diferente (por exemplo, $X=\{a,b,c,d,e,f\}$ e o conjunto $Y=\{e,c,f,b,a,d\}$) eles ainda satisfazem a propriedade de que se $t$ pertence a um deles implica $t$ pertencer ao outro. Outro ponto interessante é que não importa se um conjunto possui elementos repetidos ele continuará igual ao que possui apenas um elemento, isto é, $X=\{a,b,d,e\}$ é igual ao $Y=\{b,a,d,a,e,b,b\}$. Ou seja, em um conjunto não importa a ordem dos elementos, nem as repetições de elementos.
    
    \textbf{Axioma da Existencia do Conjunto Vazio: } Diremos que no nosso universo ($\mathbb{U}$), existe um conjunto tal que ele não contém ninguém, ou seja, ele é vazio, daí seu nome, Conjunto Vazio (denotado por $\emptyset$). O axioma é:
        
        \[\exists a \hspace{1.5mm} \forall x \hspace{1.5mm} \neg\hspace{0.5mm} (x \in a)\]
    
    \textbf{Unicidade do Conjunto Vazio}
    
    Podemos provar a unicidade do conjunto vazio a partir dos dois axiomas acimas, suponhamos que existam dois conjuntos ($A$ e $B$) com a propriedade do conjunto vazio, assim utilizando o axioma da Extensão concluíremos que eles são iguais, suponhamos $t$ arbitrário:
    
    \begin{center}
        \begin{landscape}
        \AxiomC{$\forall a  \forall b  (a = b) \iff (\forall x  (x \in a \iff x \in b))$}
        \UnaryInfC{$(A = B) \iff (\forall x  (x \in A \iff x \in B))$}
        \AxiomC{$t \in A $}
        \AxiomC{}
        \RightLabel{\scriptsize $1$}
        \UnaryInfC{$\forall x \neg (x \in A)$}
        \UnaryInfC{$\neg (t \in A)$}
        \BinaryInfC{$\perp$}
        \UnaryInfC{$t \in B$}
        \AxiomC{$t \in B$}
        \AxiomC{}
        \RightLabel{\scriptsize $1$}
        \UnaryInfC{$\forall x \neg (x \in B)$}
        \UnaryInfC{$\neg (t \in B)$}
        \BinaryInfC{$\perp$}
        \UnaryInfC{$t \in A$}
        \RightLabel{\scriptsize $\iff I_1$}
        \BinaryInfC{$t \in A \iff t \in B$}
        \UnaryInfC{$\forall x  (x \in A \iff x \in B)$}
        %\RightLabel{\scriptsize $\to E_4$}
        \BinaryInfC{A=B}
        \DisplayProof
         \end{landscape}
    \end{center}
    
    \textbf{Axioma do Par: } Este axioma nos diz que para todos os conjuntos $A$ e $B$, existe um conjunto conjunto que é $\{A,B\}$. Ou seja,
    
        \[\forall a \hspace{1.5mm} \forall b \hspace{1.5mm} \exists c \hspace{1.5mm} \forall x \hspace{1.5mm} (x \in c \leftrightarrow x = a \vee x = b\]
        
    Vale ressaltar que se $A=B$, teremos $\{A,B\}=\{A,A\}=\{A\}$. A aplicação sucessiva deste axioma nos permite criar uma infinidade de conjuntos finitos. Por exemplo, $A=\emptyset$ e seja $B=\emptyset$, assim teremos $\{\emptyset\}$, sendo $B=\{\emptyset\}$, teremos agora $\{\emptyset,\{\emptyset\}\}$, agora fazendo $A=\{\emptyset\}$, teremos $\{\{\emptyset\}\}$.
    
    Não pretendemos nos aprofundar mais nos axiomas de conjuntos, dado que eles utilizarão conceitos abordados futuramente, nos capítulos de Relações, Funções e Axiomas.
    
    Mas como se pode ver em ..., podemos reduzir uma grande gama de coisas a conjuntas, assim podemos tratá-las na Teoria dos Conjuntos. Para saber mais sobre Teoria dos Conjuntos dê uma olhada em ... .

    \section{Diagrama de Venn}
    A maneira mais simples de entender a Teoria de Conjuntos, talvez seja o Diagrama de Venn. Criado por John Venn em 1880, esse sistema de representar graficamente conjuntos auxilia imensamente quem está começando a aprender esse assunto, principalmente para entender sobre a parte inicial de notações. Basicamente, consiste em representar num plano, o universo $\mathcal U$ como sendo um retângulo e cada conjunto $A,B,...$ como uma curva fechada simples (geralmente, círculo).
    
    \subsection{Para 1 ou 2 conjuntos}
    Começando com a ideia mais simples, a imagem abaixo representa em vermelho o conjunto $A$ dentro do universo $\mathcal U$:
    
    %\begin{figure}[h!]
    %    \centering
    %    \includegraphics{fig_set_01_01.png}
    %    \caption{Conjunto $A$ dentro de $\mathcal U$}
    %    \label{fig:fig_set_01_01}
    %\end{figure}
    
    Já sobre o conjunto complementar $A^c$, ele simplesmente é a parte que está no retângulo, mas não está no círculo, justamente o que não estava de vermelho na figura anterior.
    
    %\begin{figure}[h!]
    %    \centering
    %    \includegraphics{fig_set_01_02.png}
    %    \caption{Conjunto complementar $A^c$}
    %    \label{fig:fig_set_01_02}
    %\end{figure}
    
    Para representar que um elemento pertence ao conjunto $A$, simplesmente colocamos ele dentro do espaço delimitado pelo círculo que representa o conjunto, e para representar que um elemento nāo pertence ao conjunto $A$, fazemos o inverso.
    
    
    %\begin{figure}[h!]
    %    \centering
    %    \includegraphics{figure_set_01_03.png}
    %    \caption{$a \in A$ e $b \notin A$}
    %    \label{fig:figure_set_01_03}
    %\end{figure}
    
    Quando vamos representar mais de um conjunto em um diagrama de Venn, devemos necessariamente ter todas as possíveis relações, mas o que isso significa? Por exemplo, quando temos $2$ conjuntos $A$ e $B$, significa que devemos ter $4$ regiões representando respectivamente: elementos que pertencem somente à $A$, elementos que pertencem somente à $B$, elementos que pertencem à $A$ e à $B$ simultaneamente e elementos que não pertencem a nenhum dos conjuntos. Precisamos disso, para que tudo que provarmos para dois conjuntos $A$ e $B$, possa ser generalizado para dois conjuntos quaisquer, isso será explicado melhor num exemplo posterior.
    
    Utilizando esse artífiico, podemos representar todas as definições de intersecçāo, uniāo e diferença de $2$ conjuntos, introduzidas na seçāo anterior. Veja nas figuras abaixo:
    
    %\begin{figure}[h!]
    %    \centering
    %    \includegraphics{figure_set_01_04.png}
    %    \caption{Intersecção $A \cap B$}
    %    \label{fig:figure_set_01_04}
    %\end{figure}
    
    %\begin{figure}[h!]
    %    \centering
    %    \includegraphics{figure_set_01_05.png}
    %    \caption{União $A \cup B$}
    %    \label{fig:figure_set_01_05}
    %\end{figure}
    
    %\begin{figure}[h!]
    %    \centering
    %    \includegraphics{figure_set_01_06.png}
    %    \caption{Diferença $A \setminus B$}
    %    \label{fig:figure_set_01_06}
    %\end{figure}
    
    Todavia, isso ainda não permite fazer tudo que desejamos. Se quisermos representar que $A \subseteq B$, a ideia inicial seria colocar o círculo $A$ dentro do círculo $B$, quebrando o rigor de manter todas as possíveis relações, pois não teremos uma região para representar os elementos que pertecem somente a $A$. Então, como resolver esse problema? Representamos os conjuntos $A$ e $B$ da mesma forma que anteriormente e também escrevemos o símbolo do conjunto vazio $\emptyset$ na região dos elementos que pertencem somente a $A$. Assim, só existem elementos no conjunto $A$ que estão na região $A\cap B$, ou seja, se um elemento está em $A$, como consequência ele está em $B$, exatamente a definição de $A \subseteq B$.
    
    %\begin{figure}[h!]
    %    \centering
    %    \includegraphics{figure_set_01_07.png}
    %    \caption{Subconjunto $A \subseteq B$}
    %    \label{fig:figure_set_01_07}
    %\end{figure}
    
    É inegável que para muitos exemplos isso se torna inviável, principalmente quando o único objetivo é fazer uma ilustração matemática do problema, como por exemplo: tomamos o universo $\mathcal U$ como a fauna do nosso planeta, nele temos dois conjuntos $A$ de humanos e $B$ de mamíferos. É previamente conhecido que todos os humanos são mamíferos, falando de outra forma, que $A \subseteq B$. Logo, pra representar um problema que envolva esses elementos, podemos utilizar o \textbf{Diagrama de Euler}, similar ao Diagrama de Venn, com a diferença de que não é necessário mostrar todas as possíveis relações, mas apenas as relações específicas do problema retratado. E assim, fazer exatamente o que tinha sido proposto no parágrafo anterior e, colocar o círculo $A$ dentro do círculo $B$.

    %\begin{figure}[h!]
    %    \centering
    %    \includegraphics{figure_set_01_08.png}
    %    \caption{Diagrama de Euler $A \subseteq B$}
    %    \label{fig:figure_set_01_08}
    %\end{figure}

    Já foi bastante falado sobre Diagrama de Venn, mas nem chegamos a trabalhar com mais de $2$ conjuntos, o que é importante, dado que a Teoria de Conjuntos não se resume a $A$ e $B$. Mas antes de partirmos para mais conjuntos, vamos pensar numa generalização de quantas regiões diferentes devemos ter para que o diagrama seja um Diagrama de Venn. Dado $n$ conjuntos diferentes, tomamos um elemento qualquer $x$, e para cada um dos $n$ conjuntos existem duas possibilidades: $x \in $ conjunto e $x \notin $ conjunto. Logo, concluímos que existem $\underbrace{\begin{matrix} 2\cdot2\cdots2\cdot2\end{matrix}}_{n} = 2^n$ possibilidades de pertencimento de $x$ nos conjuntos, equivalente à dizer que existem $2^n$ regiões diferentes. Isso bate perfeitamente com o caso anterior pra $2$ conjuntos, pois vimos que era necessário ter $4=2^2$ regiões diferentes.

    Agora, com $3$ conjuntos $A$, $B$ e $C$, o número de regiões diferentes é $2^3=8$, mas como iremos representá-las? A primeira ideia que vem a cabeça é adicionar um círculo representando o conjunto $C$ no diagrama da figura $\ref{fig:sets-02-00}$, intersectando as regiões já existentes, resultando na figura abaixo:

    Fazendo uma rápida contagem, obtemos $8$ regiões diferentes, exatamente como deve ser (lembrete: A região fora dos conjuntos mas dentro do universo $\mathcal U$, também é considerada na contagem). E da mesma forma que representamos bla bla bla

\section{Teoremas e Exemplos}
    
    Nesta seção iremos tratar de formalizar definições e Teoremas. Ao contrário da seção anterior, onde o foco era transmitir a notação e uma ideia geral dos conceitos.
    
    Aqui serão apresentadas algumas provas em Lean para o conteúdo apresentado. Não se preocupe, caso não entenda sobre o que o código quer dizer ignore ele por enquanto e após ler as próximas duas seções (que falam de conjuntos em Lean), volte e tente compreender o que foi feito. Já que o foco principal nesta parte são as provas matemáticas tradicionais e as definições mais precisas.
    
\subsection{Definições}

    Seja $A$ e $B$ conjuntos ($\mathbb{U}$ nosso universo e $\emptyset$ o conjunto vazio, como definidos anteriormente). Assim temos, formalmente:

    \begin{itemize}
  
        \item \textbf{Diferença}: $A \ B = \{x | x \in A \land x \notin B\}$
    
        \item \textbf{Complementar}: $\overline A = \mathbb{U} \ A = \{x | x \in \mathbb{U} \land x \notin A\}$
    \end{itemize}    

\subsection{Propriedades}
    
    A seguir teremos algumas propriedades e a prova do porque estão corretas.
    
    \begin{itemize}  
        \item $A \cup \overline A = \mathbb{U}$
    \end{itemize} 
    
        \textbf{Prova:} Seja $x$ um elemento de $A \cup \overline A$, assim temos que 
        \[x \in (A \cup \overline A)\]
        \[ \iff x \in A \vee x \in \overline A\]
        
    \begin{itemize}
        \item $A \cap \overline A = \empty{U}$
    \end{itemize} 
    
        \textbf{Prova:} Seja $x$ um elemento de $A \cap \overline A$, assim:
        
        $(i)$ Provaremos que $ \forall x (x \in A \cap \overline A) \rightarrow \forall x  (x \in \emptyset) $ :
        
        \begin{center}
            \AxiomC{}
            \RightLabel{\scriptsize $1$}
            \UnaryInfC{$ \forall x (x \in A \cap \overline A)$}
            \UnaryInfC{$t \in A \cap \overline A$}
            \UnaryInfC{$t \in A \land t \in \overline A $}
            \UnaryInfC{$t \in A $}
            \AxiomC{}
            \RightLabel{\scriptsize $1$}
            \UnaryInfC{$ \forall x (x \in A \cap \overline A)$}
            \UnaryInfC{$t \in A \cap \overline A$}
            \UnaryInfC{$t \in A \land t \in \overline A $}
            \UnaryInfC{$t \in \overline A $}
            \UnaryInfC{$t \in \mathbb{U} \land t \notin A $}
            \UnaryInfC{$t \notin A $}
            \BinaryInfC{$\perp$}
            \UnaryInfC{$t \in \emptyset$}
            \UnaryInfC{$\forall x  (x \in \emptyset)$}
            \RightLabel{\scriptsize $1$}
            \UnaryInfC{$ \forall x (x \in A \cap \overline A) \rightarrow \forall x  (x \in \emptyset) $}
            \DisplayProof
        \end{center}
        
        $(ii)$ Provaremos que $ \forall x  (x \in \emptyset) \rightarrow \forall x (x \in A \cap \overline A)$ :
        \begin{center}
            \AxiomC{}
            \RightLabel{\scriptsize $1$}
            \UnaryInfC{$\forall x  (x \in \emptyset)$}
            \UnaryInfC{$\perp$}
            \UnaryInfC{$t \in A \cap \overline A$}
            \UnaryInfC{$ \forall x (x \in A \cap \overline A)$}
            \RightLabel{\scriptsize $1$}
            \UnaryInfC{$\forall x  (x \in \emptyset) \rightarrow \forall x (x \in A \cap \overline A)$}
            \DisplayProof
        \end{center}
        
        Portanto, de $(i)$ e $(ii)$, concluímos que $ \forall x (x \in A \cap \overline A) \iff \forall x  (x \in \emptyset) $, ou seja, $A \cap \overline A$.
        
\section{Conjuntos em Lean}
    
    Ao longo deste capítulo, observamos que, embora na teoria axiomática dos conjuntos se considere conjuntos de objetos distintos, em matemática é mais comum considerar subconjuntos de algum dominio fixo ($\mathcal U $). É assim que os conjuntos são tratados no Lean. Para qualquer dado do tipo $U$, Lean nos retorna um novo dado tipo $conjunto$ $U$, que consiste nos conjuntos dos elementos de $U$. Assim, por exemplo, podemos raciocinar sobre conjuntos de números naturais, conjuntos de números inteiro ou conjuntos de pares de números naturais.
    
\subsection{Primeiros Passos}
    Dado {\fontencoding{U}\fontfamily{cmtt}\selectfont A : set U} e $x : U$, é possível escrever $x \in A$ para afirmar que $x$ é um elemento do conjunto $A$. O carácter $\in$ pode ser escrito em Lean usando $\backslash$in .

\begin{lstlisting}
import data.set
open set

variable {U : Type}
variables A B C : set U
variable x : U

#check x ∈ A
#check A ∪ B
#check B \ C
#check C ∩ A
#check -C
#check ∅ ⊆ A
#check B ⊆ univ
    
\end{lstlisting}


Abaixo temos uma pequena lista de como se representa os principais caractéres da parte de conjuntos no Lean: 

\begin{itemize}
  
    \item $\in$ $\rightarrow$ $\backslash$in
  
    \item $\notin$ $\rightarrow$ $\backslash$notin
  
    \item $\subset$ $\rightarrow$ $\backslash$subset
  
    \item $\subseteq$ $\rightarrow$ $\backslash$sub
  
    \item $\emptyset$ $\rightarrow$ $\backslash$empty
  
    \item $\cup$ $\rightarrow$ $\backslash$un \ ou \ $\backslash$cup \ ou \ $\backslash$union
  
    \item $\cap$ $\rightarrow$ $\backslash$i \ ou \ $\backslash$cap \ ou \ $\backslash$intersction

\end{itemize}

Obs$^{1}$.: O conjunto universal é denotado {\fontencoding{U}\fontfamily{cmtt}
\selectfont univ}.

Obs$^{2}$.: O complementar de um conjunto é denotada com um símbolo de negação antes de seu símbolo, assim: $-A$

$\qquad$

Noções básicas da teoria dos conjuntos são definidas na biblioteca principal do Lean, mas teoremas e notações adicionais estão disponíveis em uma biblioteca auxiliar que é carregada com o comando 
{\fontencoding{U}\fontfamily{cmtt}
\selectfont import data.set}, que deve aparecer no início do arquivo.

$\qquad$

O template abaixo serve para mostrar que o conjunto $A$ é um subconjunto de $B$:

\begin{lstlisting}
import data.set
open set

variable {U : Type}
variables A B C : set U

example : A ⊆ B :=
assume x,
assume h : x ∈ A,
show x ∈ B, from sorry

\end{lstlisting}

E o template a seguir pode ser usado para mostrar que $A$ e $B$ são iguais:

\begin{lstlisting}
import data.set
open set

variable {U : Type}
variables A B C : set U

example : A = B :=
eq_of_subset_of_subset
  (assume x,
    assume h : x ∈ A,
    show x ∈ B, from sorry)
  (assume x,
    assume h : x ∈ B,
    show x ∈ A, from sorry)
    
\end{lstlisting}

Opcionalmente, nós podemos mostrar a mesma prova de outra forma:

\begin{lstlisting}
import data.set
open set

variable {U : Type}
variables A B C : set U

example : A = B :=
ext (assume x, iff.intro
  (assume h : x ∈ A,
    show x ∈ B, from sorry)
  (assume h : x ∈ B,
    show x ∈ A, from sorry))

\end{lstlisting}

Aqui, {\fontencoding{U}\fontfamily{cmtt}
\selectfont ext} é uma sigla para ``extensionality", ou seja, extensionalidade. Matemáticamente, isso é representado desta forma:

\begin{center}
    $\forall x$ ; ($x \in A$ $\leftrightarrow$ $x \in B$) $\rightarrow$ $A = B$.  
\end{center}

Reduzindo assim a prova $A = B$ para a prova: $\forall x$ ; ($x \in A$ $\leftrightarrow$ $x \in B$), que nós podemos realizar usando $\forall$ e $\leftrightarrow$ introdução.

$\qquad$

Além disso, o Lean possui interpretação ambígua para regras de união, interseção e outras operações em conjuntos que são consideradas “definições”. Isso significa que as expressões $x$ $\in$ $A$ $\cap$ $B$ e $x$ $\in$ $A$ $\wedge$ $x$ $\in$ $B$ possuem a mesma interpretação no Lean. Isso também é válido para outras construções em conjuntos, como: $x$ $\in$ $A$ $\backslash $ $B$ e $x$ $\in$ $A$ $\wedge$ $\neg$ $(x$ $\in$ $B)$. O termo $\neg$ $(x$ $\in$ $B)$ que acabara de ser apresentado é outra forma de escrever $x$ $\notin$ $B$. Abaixo são apresentadas algumas aplicações dessa interpretação em Lean:

\begin{lstlisting}
import data.set
open set

variable {U : Type}
variables A B C : set U

example : ∀ x, x ∈ A → x ∈ B → x ∈ A ∩ B :=
assume x,
assume h₁ : x ∈ A,
assume h₂ : x ∈ B,
show x ∈ A ∩ B, from and.intro h₁ h₂
example : A ⊆ A ∪ B :=
assume x,
assume h : x ∈ A,
show x ∈ A ∪ B, from or.inl h
example : ∅ ⊆ A  :=
assume x,
assume h : x ∈ (∅ : set U),
show x ∈ A, from false.elim h

\end{lstlisting}

Observe no último exemplo a necessidade de usar a notação {\fontencoding{U}\fontfamily{cmtt}
\selectfont ($\emptyset$ : set U)}, dizendo ao Lean que o $\emptyset$ é um conjunto de {\fontencoding{U}\fontfamily{cmtt}
\selectfont (U)}. O Lean pode, em algumas situações, inferir informações como essa dentro do contexto (por exemplo, quando queremos mostrar que $x$ $\in$ $A$, onde $A$ é do tipo {\fontencoding{U}\fontfamily{cmtt}
\selectfont (set U)}), mas para isso acontecer, é necessário o uso de alguns comandos.

$\qquad$

Opcionalmente, podemos usar alguns teoremas da biblioteca do Lean, projetados especificamente para uso em conjuntos:

\begin{lstlisting}
import data.set
open set

variable {U : Type}
variables A B C : set U

example : ∀ x, x ∈ A → x ∈ B → x ∈ A ∩ B :=
assume x,
assume : x ∈ A,
assume : x ∈ B,
show x ∈ A ∩ B, from mem_inter ‹x ∈ A› ‹x ∈ B›
example : A ⊆ A ∪ B :=
assume x,
assume h : x ∈ A,
show x ∈ A ∪ B, from mem_union_left B h
example : ∅ ⊆ A  :=
assume x,
assume : x ∈ ∅,
show x ∈ A, from absurd this (not_mem_empty x)
 
\end{lstlisting}

$\qquad$

Lembre-se que o comando{\fontencoding{U}\fontfamily{cmtt}
\selectfont absurd} pode ser usado para provar qualquer fato a partir de duas hipóteses contrárias: $h_1$ : $P$ e $h_2$ : $\neg$ $P$. 

Aqui, o teorema {\fontencoding{U}\fontfamily{cmtt}
\selectfont not\_mem\_empty x} significa $x$ $\notin$ $\emptyset$. Você pode ver a declaração de teoremas disponíveis no Lean com o comando{\fontencoding{U}\fontfamily{cmtt}
\selectfont \#check}:

\begin{lstlisting}
import data.set
open set

#check @mem_inter
#check @mem_of_mem_inter_left
#check @mem_of_mem_inter_right
#check @mem_union_left
#check @mem_union_right
#check @mem_or_mem_of_mem_union
#check @not_mem_empty

\end{lstlisting}

Aqui, o símbolo{\fontencoding{U}\fontfamily{cmtt}
\selectfont @} no Lean impede que ele tente preencher argumentos implícitos automaticamente, forçando-o a exibir a declaração completa do teorema.

$\qquad$

Como o Lean pode identificar conjuntos com suas definições lógicas, auxilia a comprovação de inclusões entre conjuntos:

\begin{lstlisting}
import data.set
open set

variable {U : Type}
variables A B C : set U

example : A \ B ⊆ A :=
assume x,
assume : x ∈ A \ B,
show x ∈ A, from and.left this
example : A \ B ⊆ -B :=
assume x,
assume : x ∈ A \ B,
have x ∉ B, from and.right this,
show x ∈ -B, from this

\end{lstlisting}

Outra vez, é possível usar versões dos teoremas projetados especificamente para conjuntos:

\begin{lstlisting}
import data.set
open set

variable {U : Type}
variables A B C : set U

example : A \ B ⊆ A :=
assume x,
assume : x ∈ A \ B,
show x ∈ A, from mem_of_mem_diff this

example : A \ B ⊆ -B :=
assume x,
assume : x ∈ A \ B,
have x ∉ B, from not_mem_of_mem_diff this,
show x ∈ -B, from this

\end{lstlisting}

$\qquad$

Como o Lean tem que desenvolver definições, ele pode acabar se confundindo às vezes. Por exemplo, na prova a seguir, se você subtituir a última linha por{\fontencoding{U}\fontfamily{cmtt}
\selectfont sorry}, o Lean terá problemas tentando entender que você quer que ele desenvolva o símbolo de subconjunto:

\begin{lstlisting}
variable  {U : Type}
variables A B : set U
example : A ∩ B ⊆ B ∩ A :=
assume x,
assume h : x ∈ A ∩ B,
have h₁ : x ∈ A, from and.left h,
have h₂ : x ∈ B, from and.right h,
and.intro h₂ h₁

\end{lstlisting}

$\qquad$

Uma solução alternativa é usar o comando{\fontencoding{U}\fontfamily{cmtt}
\selectfont show}. Na maioria das vezes, fornecer informações adicionais para o Lean pode ser útil. Outra solução é nomear um teorema, o que leva o Lean a usar um método um pouco diferente de processar a prova, corrigindo o problema como um efeito colateral de sorte.

\begin{lstlisting}
variable  {U : Type}
variables A B : set U

example : A ∩ B ⊆ B ∩ A :=
assume x,
assume h : x ∈ A ∩ B,
have h₁ : x ∈ A, from and.left h,
have h₂ : x ∈ B, from and.right h,
show x ∈ B ∩ A, from sorry

theorem my_example : A ∩ B ⊆ B ∩ A :=
assume x,
assume h : x ∈ A ∩ B,
have h₁ : x ∈ A, from and.left h,
have h₂ : x ∈ B, from and.right h,
sorry

\end{lstlisting}

\subsection{Identidades de conjuntos}

Nessa seção, falaremos brevemente sobre Identidades de conjuntos.

Iniciaremos com um exemplo simples: a regra da distributividade para conjuntos: $A \cup (B \cap C)$ = $(A \cup B) \cap (A \cup C)$. Uma maneira de prova-la é:


\begin{lstlisting}
import data.set
open set

variable {U : Type}
variables A B C : set U

example : A ∪ (B ∩ C) = (A ∪ B) ∩ (A ∪ C) :=
eq_of_subset_of_subset
  (assume x,
    assume h : x ∈ A ∪ (B ∩ C),
    or.elim h
      (assume h₁ : x ∈ A,
        have h₂ : x ∈ A ∪ B, from or.inl h₁,
        have h₃ : x ∈ A ∪ C, from or.inl h₁,
        show x ∈ (A ∪ B) ∩ (A ∪ C), from and.intro h₂ h₃)
      (assume h₁ : x ∈ B ∩ C,
        have h₂ : x ∈ B, from and.left h₁,
        have h₃ : x ∈ C, from and.right h₁,
        have h₄ : x ∈ A ∪ B, from or.inr h₂,
        have h₅ : x ∈ A ∪ C, from or.inr h₃,
        show x ∈ (A ∪ B) ∩ (A ∪ C), from and.intro h₄ h₅))
  (assume x,
    assume h : x ∈ (A ∪ B) ∩ (A ∪ C),
    have h₁ : x ∈ A ∪ B, from and.left h,
    have h₂ : x ∈ A ∪ C, from and.right h,
    or.elim h₁
      (assume h₃ : x ∈ A,
        show x ∈ A ∪ (B ∩ C), from or.inl h₃)
      (assume h₃ : x ∈ B,
      or.elim h₂
        (assume h₄ : x ∈ A,
        show x ∈ A ∪ (B ∩ C), from or.inl h₄)
        (assume h₄ : x ∈ C,
        have h₅ : x ∈ B ∩ C, from and.intro h₃ h₄, 
        show x ∈ A ∪ (B ∩ C), from or.inr h₅)))
\end{lstlisting}

Outra identidade que usaremos como exemplo aqui é a Lei de Absorção:  

\begin{center}
    $A \cup (A \cap B) = A$
\end{center}

Nos templates abaixo temos duas formas de escrever a prova: a que estamos utilizando ao longo deste capítulo; e em forma de lógica booleana, trocando o sinal de igual pelo de equivalência, ou ``se e somente se", ($\leftrightarrow$). (Você consegue reprentá-lo no Lean escrevendo: $\backslash$iff ou $\backslash$lr)

\begin{lstlisting}
import data.set
open set

theorem inter_subseq (H : Type)(P Q : set H) : P ∩ (P ∪ Q) = P :=
eq_of_subset_of_subset
  (assume x,
    assume h : x ∈ P ∩ (P ∪ Q),
    show x ∈ P, from h.left)
  (assume x,
    assume h : x ∈ P,
    have h₁ : x ∈ P ∪ Q, from or.inl h,
    show x ∈ P ∩ (P ∪ Q), from and.intro h h₁)

variable  U : Type
variables A B : set U

example : A ∪ (A ∩ B) = A :=
calc
  A ∪ (A ∩ B) = (A ∪ A) ∩ (A ∪ B) : by rw union_distrib_left
           ... = A ∩ (A ∪ B)       : by rw union_self
           ... = A                 : by rw inter_subseq
\end{lstlisting}

\begin{lstlisting}
import logic.basic
open classical

theorem left_of_and (P Q : Prop) : P ∧ (P ∨ Q) ↔ P :=
iff.intro
  (assume h : P ∧ (P ∨ Q),
  show P, from h.left)
  (assume h : P,
  have h₁ : P ∨ Q, from or.inl h,
  show P ∧ (P ∨ Q), from and.intro h h₁)

variables A B : Prop

example : A ∨ (A ∧ B) ↔ A :=
calc
  A ∨ (A ∧ B) ↔ (A ∨ A) ∧ (A ∨ B) : by rw or_and_distrib_left
          ... ↔ A ∧ (A ∨ B)        : by rw or_self
          ... ↔ A                  : by rw left_of_and
\end{lstlisting}

A função calc, já apresentada em outras frentes desse livro, aparece novamente, agora para conjuntos. A próxima seção explanará como aplicar calc em conjuntos no Lean e mostrará a diferença entre resolver questões da forma clássica e com calc.

\section{Aplicação de calc em Conjuntos}

\section{Famílias Indexadas}

\section{Power Set}

\section{Exercícios}

\begin{enumerate}
    
\item Questões de Identidades de conjuntos
    
\begin{lstlisting}
import data.set
open set

section
  variable U : Type
  variables A B C : set U

--Comutatividade em ∩ e ∪ 
  example : A ∩ B = B ∩ A := 
  eq_of_subset_of_subset
    (assume x,
        assume h : x ∈ A ∩ B,
        have h₁ : x ∈ A, from h.left,
        have h₂ : x ∈ B, from h.right,
        show x ∈ B ∩ A, from and.intro h₂ h₁)
    (assume x,
        assume h : x ∈ B ∩ A,
        have h₁ : x ∈ B, from h.left,
        have h₂ : x ∈ A, from h.right,
        show x ∈ A ∩ B, from and.intro h₂ h₁)

  example : A ∪ B = B ∪ A:=
  eq_of_subset_of_subset
    (assume x,
        assume h : x ∈ A ∪ B,
        or.elim h
        (assume h₁ : x ∈ A,
        show x ∈ B ∪ A, from or.inr h₁)
        (assume h₁ : x ∈ B,
        show x ∈ B ∪ A, from or.inl h₁))
    (assume x,
        assume h : x ∈ B ∪ A,
        or.elim h
        (assume h₁ : x ∈ B,
        show x ∈ A ∪ B, from or.inr h₁)
        (assume h₁ : x ∈ A,
        show x ∈ A ∪ B, from or.inl h₁))

--Associatividade
  example : (A ∩ B) ∩ C = A ∩ (B ∩ C) :=
  eq_of_subset_of_subset
    (assume x,
        assume h : x ∈ (A ∩ B) ∩ C,
        have h₁ : x ∈ A ∩ B, from h.left,
        have h₂ : x ∈ B ∩ C, from and.intro h₁.right h.right,
        show x ∈ A ∩ (B ∩ C), from and.intro h₁.left h₂) 
    (assume x,
        assume h : x ∈ A ∩ (B ∩ C),
        have h₁ : x ∈ B ∩ C, from h.right,
        have h₂ : x ∈ A ∩ B, from and.intro h.left h₁.left,
        show x ∈ (A ∩ B) ∩ C, from and.intro h₂ h₁.right) 

--De Morgan
  example : A \ (B ∩ C) = (A \ B) ∪ (A \ C) := sorry
end
\end{lstlisting}

\item Prove que $A \cup \overline A = \mathcal U$. (Fornecendo uma prova tradicional e uma em Lean.)

\textbf{Resposta} 




\item Questões com calc
    
    
\end{enumerate}
